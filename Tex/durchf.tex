\section{Execution}
The first step is to set up the apparatus according to Figure \ref{fig:aufbau}.
The superheterodyne receiver is adjusted by slowly increasing the ZF-amplifier until a voltage is visible.
The preamplifier is then set up so that the output voltage becomes maximum.
Then the bridge is adjusted. Then this process is repeated until the maximum ZF-gain is reached.
\begin{figure}
  \centering
  \begin{circuitikz}
    \draw[fill=black] (0,0) rectangle (0.5,2);
    \draw[fill=black] (3.5,0) rectangle (4,2);
    \draw (0,2) to[R] (-2.5,2);
    \draw (-2.5,2) rectangle (-3.5,1);
    \node at (-3,1.5){A};
    \draw (-3.5,1.9)--(-4,1.9) to[esource] (-4,1.1)--(-3.5,1.1);
    \node at (-4,1.5){B};
    \draw (-2,2.7) to[voltmeter] (-0.5,2.7)--(-0.5,1.9)[-*];
    \draw (-2,2.7)--(-2,1.9)[-*];
    \draw (0,0)--(-0.3,0)--(-0.3,1.5)--(-2.5,1.5);
    \draw (1.75,0) to[cute inductor,L=$L$] (1.75,1.5);
    \draw[fill=black] (1.65,0.4) rectangle (1.8,1.1);
    \draw (1.75,1.5)--(2,1.5)--(2,0);
    \draw (1.75,0)--(1.75,-1) to[vR] (0.5,-1)--(0.5, -2);
    \draw (3,-2) to[cute inductor] (1.75,-2) to[cute inductor] (0.5,-2);
    \draw (3,-1) to[variable capacitor] (1.75,-1);
    \draw (3,-1)--(3,-0.5)--(2,-0.5)--(2,0);
    \draw[fill=black] (2.5,-2.35) rectangle (1,-2.4);
    \draw (0.7,-2.65) to[cute inductor] (2.8,-2.65);
    \draw (0.7,-2.65) --(0.7,-3);
    \draw (2.8,-2.65)--(2.8,-3);
    \draw (0.5,-3) rectangle (3,-3.6);
    \node at (1.75,-3.3){C};
    \draw (3,-2) -- (3,-1);
    \draw (1.75,-2)--(1.75,-1.8)-- (4,-1.8);
    \draw (1.75,-1) -- (1.75,-1.6)--(4,-1.6);
    \draw (4,-1.6) node[ground,rotate=180]{} (4,-1.6);
    \draw (4,-1.6)--(4.4,-1.6);
    \draw (4,-1.8)--(4.4,-1.8);
    \draw (4.4,-2)rectangle(4.8,-1.4);
    \node at (4.6,-1.65){D};
    \draw (4.8,-1.6)--(5.5,-1.6);
    \draw (4.8,-1.8)--(4.9,-1.8)--(4.9,-2.4)--(5.5,-2.4);
    \draw (5.5,-1.6) to[voltmeter] (5.5,-2.4);
    \draw[-*] (1.75,-0.9)--(1.75,-1.1);
    \draw[-*] (3,-0.9)--(3,-1.1);
    \draw[-*] (1.75,-2)--(1.75,-2.1);
    \draw[*-] (4.9,-2.3)--(4.9,-4.5);
    \draw[*-] (5,-1.5)--(5,-5);
    \draw (2,-5)--(5,-5);
    \draw (2,-4.5)--(4.9,-4.5);
    \draw (2,-5.2)rectangle(0,-4.3);
    \node at (2.7,-4.75){Y input};
    \node at (1,-4.6){F};
    \draw (-0.5,1.9)--(-0.5,-4.5)--(0,-4.5);
    \draw (-2,1.9)--(-2,-5)--(0,-5);
    \node at (-0.7,-4.75){X input};
    \node[anchor=west] at (4.5,2){Legend};
    \node[anchor=west] at (4.5,1.7){A=Power amplifier};
    \node[anchor=west] at (4.5,1.4){B=Ramp generator};
    \node[anchor=west] at (4.5,1.1){C=Crystal oscillator};
    \node[anchor=west] at (4.5,0.8){D=Superheterodyne receiver};
    \node[anchor=west] at (4.5,0.5){F=XY recorder};
  \end{circuitikz}
  \caption{Measurement setup of the electron spin resonance method.
  The Helmholtz coil generates an external field proportional to the fed-in ramp voltage.
  The change of the complex resistance of the coil with the sample leads to a change of the bridge voltage,
  which is fed with a high frequency voltage.}
  \label{fig:aufbau}
\end{figure}
Then the ramp generator is switched on. A resonance curve should now be visible, which is recorded by the XY recorder.
The required current for the resonance can be read from the drawing and thus the gyromagnetic ratio of the sample can be determined
according to the formula \ref{eq:gyro}. The magnetic field can be determined with formula \ref{eq:mfeld}.
