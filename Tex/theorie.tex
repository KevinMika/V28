\section{Motivation}
The motivation of this Experiment is to determine the magnetic moment caused by the Spin of a free electron.
Diphenylpikrylhydrazyl is a sample which offers free electrons. Methods of high frequency spectroscopy are used for this purpose:
The experimental conditions are varied until a resonance absorption of the electrons occur. With the resonance condition it is possible
to determine the magnetic moment.
\section{Theory}
Electrons without orbital angular momentum still have a magnetic moment.
From this it can be concluded that the electrons have a spin. The connection between orbital
angular momentum and magnetic momentum in
of quantum mechanics, based on the consideration that the
wave function for an atom in the one-electron approximation can be represented as
\begin{equation}
  \psi_{n,l,m}(r,\vartheta,\varphi)=R_{n,l}(r)\uptheta_{l,m}(\vartheta)\upphi(\varphi)=\frac{R_{n,l}(r)\uptheta_{l,m}\symup{e}^{im\varphi}}{\sqrt{2\pi}},
\end{equation}
leading to the expression
\begin{equation}
  \mu_z=-\frac{e_0\hbar m}{2m_0}
\end{equation}
which describes the relationship between magnetic
and the angular momentum. $m\hbar$ is the angular momentum $l$ and the product of the natural constants is called Bohr magneton $\mu_B$.
