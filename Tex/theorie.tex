\section{Motivation}
The motivation of this Experiment is to determine the magnetic moment caused by the Spin of a free electron.
Diphenylpikrylhydrazyl is a sample which offers free electrons. Methods of high frequency spectroscopy are used for this purpose:
The experimental conditions are varied until a resonance absorption of the electrons occur. With the resonance condition it is possible
to determine the magnetic moment.
\section{Theory}
Electrons without orbital angular momentum still have a magnetic moment.
From this it can be concluded that the electrons have a spin. The connection between orbital
angular momentum and magnetic momentum in
of quantum mechanics, based on the consideration that the
wave function for an atom in the one-electron approximation can be represented as
\begin{equation}
  \psi_{n,l,m}(r,\vartheta,\varphi)=R_{n,l}(r)\uptheta_{l,m}(\vartheta)\upphi(\varphi)=\frac{R_{n,l}(r)\uptheta_{l,m}\symup{e}^{im\varphi}}{\sqrt{2\pi}},
\end{equation}
leading to the expression
\begin{equation}
  \mu_B=-\frac{e_0\hbar m}{2m_0}
\end{equation}
which describes the relationship between magnetic
and the angular momentum. $m\hbar$ is the angular momentum $l$ and the product of the natural constants is called Bohr magneton $\mu_B$.
\subsection{Spin relative to an external magnetic field}
If an external magnetic field is applied, the energy levels are split (Zeeman effect).
In addition, a force
\begin{equation}
  F_Z=\mu_{S,Z}\frac{\partial B_Z}{\partial Z}
\end{equation}
acts on the magnetic moments, thus deflecting the beam.
depending on the orientation of the moment relative
to the external magnetic field. The spin quantum number is calculated as
then according to
\begin{equation}
  2s+1=2
\end{equation}
This is due to the fact that the components of a vector are at most equal to their
amount, the Orientation quantum number $m$ only assume the following values:
\begin{equation}
  m=0,\pm1,\pm2,...,\pm l
\end{equation}
So there are only $2l+1$ settings for the Orientation quantum number $m$, where $l$ is the angular momentum.
Therefore, the following relationship applies to the Z-component of the spin:
\begin{equation}
  S_Z=m_s\hbar=\pm\frac{\hbar}{2}.
\end{equation}
The magnetic moment $\mu$ of the spin is then
\begin{equation}
  \mu_{S_Z}=-\frac{g\cdot\mu_B}{2}.
\end{equation}
$g$ is referred to as the gyromagnetic ratio and may assume a value other than 1.
\subsection{Electron paramagnetic resonance}
The electron paramagnetic resonance method is used to determine the gyromagnetic ratio.
With an external magnetic field, the simple energy levels are converted into two sub-levels,
where the energy difference between the two levels is
\begin{equation}
  \increment E=g\mu_B B.
  \label{eq:ediff}
\end{equation}
If now energy, which corresponds to the energy difference,
is supplied to the system in form of light quanta, then \ref{eq:ediff} results to
\begin{equation}
  h\nu=g\mu_B B.
\end{equation}
The electrons are now able to go into the higher energy state and their spin is reversed.
\begin{figure}
  \centering
  \begin{circuitikz}
    \draw[->,brown] (1,2)--(1,0);
    \draw[->,brown] (2,2)--(2,0);
    \draw[->,brown] (3,2)--(3,0);
    \node at (3.3,0){$H$};
    \draw[->] (-1,0)--(-1,2);
    \draw (-1.1,1.5)--(-0.9,1.5);
    \draw (-1.1,0.7)--(-0.9,0.7);
    \node at (-1.3,2.3){$E$};
    \node[font=\small] at (-1.4,1.5){$E_{\downarrow}$};
    \node[font=\small] at (-1.4,0.7){$E_{\uparrow}$};
    \draw[->] (1,0.7)--(1.2,0.9);
    \draw[->] (1,1.5)--(1.2,1.3);
    \draw[decorate,decoration={brace,amplitude=6pt}](-1,1.5)--(-1,0.7);
    \node[font=\small] at (-0.1,1.1){$\increment E=h\nu$};
  \end{circuitikz}
  \caption{Energy levels $E_{\downarrow}$ and $E_{\uparrow}$ split by an external magnetic field $H$. $\increment E$ can be modified by varying "H".}
  \label{}
\end{figure}
The sample to be investigated is wrapped in a coil which is connected to a bridge circuit and placed in a Helmholtz coil.
The magnetic field of a Helmholtz coil can be determined with
\begin{equation}
  B(I)=\frac{8\mu_0 n I}{\sqrt{125}\cdot r}
\end{equation}
where $\mu_0$ is the Vacuum permeability. A high-frequency voltage is applied to the bridge circuit.
The coil in which the sample is located serves to supply energy,
while the Helmholtz coil generates the external, homogeneous magnetic field.
